\documentclass[a4paper]{siamart220329}

\usepackage{damacros}

% Theorem environment for questions: Small-caps header, italize body.
\theoremstyle{plain}
% \theoremheaderfont{\normalfont\sc}
\theoremheaderfont{\normalfont\bf}
\theorembodyfont{\normalfont}
\theoremseparator{.}
\theoremsymbol{}
\newtheorem{question}{Question}
\crefname{question}{Question}{Questions}
\Crefname{question}{Question}{Questions}
\renewcommand{\phi}{\varphi}

\usepackage{listings}
\definecolor{LightGrey}{rgb}{0.9629411,0.9629411,0.9629411}
\definecolor{LighterGrey}{gray}{0.99}
\definecolor{Mauve}{rgb}{0.58,0,0.82}
\definecolor{Emerald}{rgb}{0.31, 0.78, 0.47}
\definecolor{RoyalBlue}{rgb}{0.25, 0.41, 0.88}
\definecolor{myGreen}{cmyk}{0.82,0.11,1,0.25}
\lstset{
    language=Matlab,
    keywordstyle=\color{RoyalBlue},
    basicstyle=\ttfamily,
    morekeywords={define,ifndef,endif},
    commentstyle=\color{myGreen}\small\ttfamily,
    %directivestyle=\color{Mauve}\scriptsize\ttfamily,
    showspaces=false,            
    showstringspaces=false,
    stringstyle=\color{Mauve}\ttfamily,
    numbers=left,
    numberstyle=\scriptsize,
    stepnumber=1,
    numbersep=8pt,
    showstringspaces=false,
    breaklines=true,
    frameround=ftff,
    frame=lines,
    backgroundcolor=\color{LightGrey}
} 
% \usepackage{xr}
% \externaldocument[T1-]{../../../Week1/Tutorial/Questions/tutorial-01}

% Sets running headers as well as PDF title and authors
\headers{Tutorial 2}{D. Avitabile}

% Title. If the supplement option is on, then "Supplementary Material"
% is automatically inserted before the title.
\title{Nonlinear Dynamical Systems Part 3 \\ 
  Dynamics in pattern-forming systems \\
  Tutorial 3
} 

% Authors: full names plus addresses.
\author{%
  Daniele Avitabile%
  \thanks{%
    Vrije Universiteit Amsterdam,
    Department of Mathematics,
    Faculteit der Exacte Wetenschappen,
    De Boelelaan 1081a,
    1081 HV Amsterdam, The Netherlands.
  \protect\\
    Inria Sophia Antipolis M\'editerran\'ee Research Centre,
    MathNeuro Team,
    2004 route des Lucioles-Boîte Postale 93 06902,
    Sophia Antipolis, Cedex, France.
  \protect\\
    (\email{d.avitabile@vu.nl}, \url{www.danieleavitabile.com}).
  }
 % \and
%   Paul T. Frank \thanks{Department of Applied Mathematics, Fictional University, Boise, ID 
% (\email{ptfrank@fictional.edu}, \email{jesmith@fictional.edu}).}
% \and Jane E. Smith\footnotemark[3]
}

\begin{document}

\maketitle

\begin{abstract}
  Themes of this tutorial:
  \begin{enumerate}
    \item Working with systems of PDEs
    \item Determining dispersion relations
    \item Determining Hopf and Turing bifurcations
    \item Predicting wavelengths and/or temporal periods of patterns near Hopf and
      Turing bifurcations.
  \end{enumerate}
\end{abstract}

\section{Introduction}
  Let us consider the following system of PDEs, known as the Schnakenberg model,
  \begin{equation}\label{eq:Sch}
    \begin{aligned}
      & \partial_t u =   \partial_x^2 u - u + u^2 v, 
	&& (x,t) \in \RSet/ 2L \ZSet \times \RSet_{>0}, \\
      & \partial_t v = d \partial_x^2 v + b - u^2v,
	&& (x,t) \in \RSet/ 2L \ZSet \times \RSet_{>0}.\\
    \end{aligned}
  \end{equation} 
  where $d,b \in \RSet_{>0}$. By writing $x \in \RSet / 2L \ZSet$ we compactly state that we are interested in
  $2L$-periodic functions: we can consider the equation posed on $[-L,L)$, or
  $[0,2L)$, or $[nL,(n+2)L)$ with $n \in \ZSet$, subject to
  \textit{periodic boundary conditions} (PBCs). Initial conditions to the problem
  will be specified in due course.

\begin{question}[Homogeneous steady state]\label{ques:HSS}
  Show that  for any $d,b >0$ system \cref{eq:Sch} admits a unique homogeneous
  equilibrium $(u_*,v_*)$, independent of $d$. Express $(u_*,v_*)$ in terms of $b$.
\end{question}

\begin{question}[Linearisation]
  Show that small perturbations $\varphi = (\tilde u, \tilde v)$ around $(u_*,v_*)$ evolve
  according to a linear PDE
  \begin{equation}\label{eq:linear}
    \partial_t \phi = \calL(u_*,v_*) \phi, 
    \qquad \text{or, in extended form,} \qquad
    \partial_t 
  \begin{pmatrix} \tilde u \\ \tilde v \end{pmatrix} 
  = \calL(u_*,v_*)
  \begin{pmatrix} \tilde u \\ \tilde v \end{pmatrix} 
  ,\end{equation} 
  and, using the result in \cref{ques:HSS}, express $\calL$ in terms of $b$ and $d$. 
  %Henceforth, with a small abuse of notation, we shall use
  %$\calL(u_*,v_*)$ and $\calL(b,d)$ interchangeably.
\end{question}

\begin{question}[Dispersion relation]\label{ques:eigenvals}
  Let us study perturbations to the homogeneous equilibrium disregarding
  boundary conditions. Loosely speaking, one way to achieve that is to think of the
  problem \cref{eq:Sch} with $x \in \RSet$ as opposed to $ x \in \RSet/2L\ZSet$. Show
  that the linear problem \cref{eq:linear}, posed on $\RSet$ admits solutions in the
  form 
  \[
    \phi(x,t) = \psi \exp(\lambda t + ik x) + \overline{\psi \exp(\lambda t + i k x)}, \qquad \psi \in \CSet^2,
  \] 
  where the overbar denotes complex conjugation, provided $\lambda, b, d, k$ satisfy
  the dispersion relation
  \begin{equation} \label{eq:dispersion}
    \lambda^2 - \tau(k,b,d)\lambda + \Delta(k,b,d) = 0,
  \end{equation}
  where $\tau$ and $\Delta$ are trace and determinant of the matrix
  \[
    M(k,b,d) = 
    \begin{pmatrix} 1-k^2 && b^2 \\ -2 && -dk^2 - b^2 \end{pmatrix},
  \]
  respectively, and $(\lambda,\psi)$ is an eigenpair of $M(k,b,d)$.
\end{question}

\begin{question}[Curves of eigenvalues] \label{question:lambdaCurves}
  As we have seen in the lectures, we
  investigate linear stability of the homogeneous steady states as follows: let
  $(u_*,v_*)$ be a homogeneous steady state of \cref{eq:Sch} posed on $x \in \RSet$ for $(b,d) = (b_*,d_*)$ and
  let $\gamma_{1,2}$ be the curves of eigenvalues
  \[
    \gamma_{1,2} \colon \RSet \to \CSet, \qquad k \mapsto \lambda_{1,2}(k,b_*,d_*).
  \] 
  It is sometimes useful to omit the dependence on the parameters $(b_*,d_*)$, and
  use $\lambda_{1,2}(k)$ and $\gamma_{1,2}(k)$ interchangeably. The steady state
  $(u_*,v_*)$ is linearly stable if  $\real \gamma_{1,2}(k) <0$, that is, if the
  eigenvalues $\lambda_{1,2}(k)$ obtained for fixed $(b_*,d_*)$ have negative real
  parts for all wavelengths $k \in \RSet$.

  Prove that that the homogeneous steady state $(u_*, v_*) \equiv (3,1/3)$ to
  \cref{eq:Sch} posed on $\RSet$ is linearly stable for $(b_*,d_*) = (3,1)$.

\end{question}
\begin{question}[Plot eigenvalues]\label{ques:plot} 
  Write code that, for fixed value $(b_*,d_*)$, plots graphs of the curves
  \[
  \real \gamma_1(k), \qquad \imag \gamma_1(k), \qquad \real \gamma_2(k), \qquad \imag
  \gamma_2(k),
  \] 
  and test the code by verifying the analytical prediction of
  \cref{question:lambdaCurves}.
\end{question}

\begin{question}[Periodic Boundary conditions] 
  As in any system of PDEs, boundary conditions have an impact on the problem.
  Discuss with a colleague the following argument, and fill the gaps where necessary.
  %(note that we do not expect you to produce a proof, but a formal argument will
  %suffice).

  Let us consider problem \cref{eq:Sch}, which has now the correct periodic boundary
  conditions, and consider the linearised problem
  \begin{equation}\label{eq:linearBCs}
    \partial_t \phi(x,t) = \calL(u_*,v_*) \phi(x,t), \qquad
	(x,t) \in \RSet/ 2L \ZSet \times \RSet_{>0}, 
  \end{equation} 
  Small perturbations $\phi = (\tilde u, \tilde v)$ to a homogeneous
  steady state $(u_*,v_*)$ of \cref{eq:Sch} with $(b,d) = (b_*,d_*)$ can be written
  \[
    \phi(x,t) = \sum_{j \in \ZSet}  \psi_j \exp (\lambda_{j} t + i k_j x) +
    \overline{\psi_j \exp (\lambda_j t + i k_j x)}
  \]
  where
  \[
  k_j = j \frac{\pi}{L}, \qquad \lambda_j^2 - \tau(k_j,b_*,d_*) \lambda_j +
  \Delta(k_j,b_*,d_*) = 0, \qquad j \in \ZSet.
\] 
You may assume, without proof, that the series above converges to $\phi$.
\end{question}

\begin{question}[Hopf bifurcation]\label{ques:Hopf}
  Using the code in \cref{ques:plot}, show that the steady state $(u_*,v_*)
  = (0.99,1/99)$ is linearly unstable when $(b_*,d_*) = (0.99,1)$. With these data,
  argue that $(u_*,v_*)$ is close to a Hopf bifurcation. What is the spatial
  wavelength and/or temporal period of the emerging pattern?
\end{question}

\begin{question}[Hopf bifurcation]\label{ques:HopfTimeStep}
  Produce numerical evidence in support of your predictions in \cref{ques:Hopf}. One
  way to do this is to perform a time simulation of \cref{eq:Sch} for $(b_*,d_*)$
  fixed as in \cref{ques:Hopf}, and with initial condition $(u(x,0),v(x,0)) =
  (u_*,v_*) + ( \tilde u(x), \tilde v(x))$, for ``small" $(\tilde u, \tilde v)$. The
  analysis in \cref{ques:Hopf} can not determine whether the bifurcation is sub- or
  super-critical. However, if it is supercritical, simulations should show a pattern
  with features predicted by the eigenvalues and eigenfunctions determined in
  \cref{ques:Hopf}.

  Using a suitable differentiation matrix $D_{xx}$ (see Tutorials 1 and 2),
  discretise \cref{eq:Sch} as a set of ODEs of the type
  \[
  \begin{pmatrix} \dot U \\ \dot V \end{pmatrix} 
  = 
  \begin{pmatrix} D_{x x} && 0 \\ 0 && d D_{xx}\end{pmatrix} 
  \begin{pmatrix}  U \\  V \end{pmatrix}  + 
  \begin{pmatrix}  f(U,V) \\  g(U,V,b) \end{pmatrix},\] 
where $U, V \in \RSet^n$, $D_{xx} \in \RSet^{n \times n}$, 
\begin{equation}\label{eq:discretised}
  \begin{aligned}
  & f \colon \RSet^n \times \RSet^n \to \RSet^n, 
    && ( f(U,V) )_i = -u_i + u_i^2 v_i, && i = 1,\ldots,n, \\
  & g \colon \RSet^n \times \RSet^n \times \RSet_{\geq 0} \to \RSet^n, 
    && ( g(U,V,b))_i = b- u_i^2 v_i, && i = 1,\ldots,n.
  \end{aligned}
\end{equation}
%
Time step the problem using $1501$ evenly distributed grid points $\{ x_j \} \subset
[-L,L]$ with $L=30$, and periodic boundary conditions and initial condition
\[
  \begin{pmatrix} u(x,0) \\ v(x,0) \end{pmatrix} = 
  \begin{pmatrix} u_* \\ v_* \end{pmatrix} +
  0.01
  \begin{pmatrix} \sin(\pi x/5) \\ \sin(\pi x/10) \end{pmatrix}.
\]

The time stepper \lstinline|ode15s| or \lstinline|ode23s| will perform better if you
pass not only the right-hand side
of \cref{eq:discretised}, but also its Jacobian with respect to $(U,V)$ (this is done
by passing the option \lstinline|'Jacobian'| to the time stepper, via Matlab's command \lstinline|odeset|).
\end{question}

Produce numerical evidence that the model exhibits a pattern compatible with your
analysis in \cref{ques:Hopf}, and in particular that the period of the emerging
oscillations approximates the one predicted by your analysis. Note that the analysis
predicts the period of the spatially continuous problem, whereas the numerical
simulation timesteps an approximation to this system with $n=1501$ grid points, hence
you should expect some discrepancy between the predicted and the observed period.

\begin{question}[Turing bifurcation]\label{ques:Turing}
  Using the code in \cref{ques:plot}, show that the steady state $(u_*,v_*)
  = (2,1/2)$ is linearly unstable when $(b_*,d_*) = (2,30)$. With these data,
  argue that $(u_*,v_*)$ is close to a Turing bifurcation. What are the wavelengths
  of the unstable modes when $L=30$? Which wavelength would you expect to observe in a
  time simulation with $L=30$?
\end{question}

\begin{question}[Turing bifurcation simulation]\label{ques:TuringTimeStep}
Produce numerical evidence that the model exhibits a pattern compatible with your
analysis in \cref{ques:Turing}, and in particular that the wavelength of the emerging
pattern matches the one predicted by your analysis. Use an initial condition of the
type
\[
  \begin{pmatrix} u(x,0) \\ v(x,0) \end{pmatrix} = 
  \begin{pmatrix} u_* \\ v_* \end{pmatrix} +
  \eps
  \begin{pmatrix} \xi(x) \\ \eta(x) \end{pmatrix},
\]
where $\xi(x)$ and $\eta(x)$ should also be selected using information from
\cref{ques:Turing}. Some fiddling with $\eps$ may be necessary to see the pattern.

\end{question}



% \bibliographystyle{siamplain}
% \bibliography{references}
\end{document}
